\section{ Tri : Algorithmes et Utilisation}

\subsection{ Introduction au Tri}
Le tri consiste à organiser les éléments d'un tableau ou d'une liste dans un ordre spécifique (croissant ou décroissant). Voici quelques algorithmes de tri couramment utilisés et leurs caractéristiques.

---

\subsection{ Tri à Bulles (Bubble Sort)}
Le tri à bulles compare chaque paire d'éléments adjacents et les échange s'ils sont dans le mauvais ordre. Cet algorithme est simple mais inefficace pour les grands tableaux.

\textbf{Exemple :}
\begin{tcolorbox}[colframe=blue!50!black, colback=blue!5!white, title=Exemple d'Attributs et Méthodes Statistiques]
\begin{verbatim}
void bubbleSort(int arr[], int n) {
    for (int i = 0; i < n - 1; i++) {
        for (int j = 0; j < n - i - 1; j++) {
            if (arr[j] > arr[j + 1]) {
                // Échange
                int temp = arr[j];
                arr[j] = arr[j + 1];
                arr[j + 1] = temp;
            }
        }
    }
}
\end{verbatim}
\end{tcolorbox}

\textbf{Complexité :}
\begin{itemize}
    \item Meilleur cas : $O(n)$ (tableau déjà trié).
    \item Pire cas : $O(n^2)$ (tableau trié dans l'ordre inverse).
    \item Moyenne : $O(n^2)$.
\end{itemize}

---

\subsection{ Tri par Insertion (Insertion Sort)}
Le tri par insertion insère chaque élément à sa position correcte dans un sous-tableau déjà trié.

\textbf{Exemple :}
\begin{tcolorbox}[colframe=blue!50!black, colback=blue!5!white, title=Exemple d'Attributs et Méthodes Statistiques]
\begin{verbatim}
void insertionSort(int arr[], int n) {
    for (int i = 1; i < n; i++) {
        int key = arr[i];
        int j = i - 1;

        // Déplace les éléments plus grands que key
        while (j >= 0 && arr[j] > key) {
            arr[j + 1] = arr[j];
            j--;
        }
        arr[j + 1] = key;
    }
}
\end{verbatim}
\end{tcolorbox}

\textbf{Complexité :}
\begin{itemize}
    \item Meilleur cas : $O(n)$.
    \item Pire cas : $O(n^2)$.
    \item Moyenne : $O(n^2)$.
\end{itemize}

---

\subsection{ Utilisation de \texttt{std::sort}}
La bibliothèque standard C++ propose l'algorithme \texttt{std::sort}, qui est beaucoup plus rapide et optimisé.

\textbf{Exemple :}
\begin{tcolorbox}[colframe=blue!50!black, colback=blue!5!white, title=Exemple d'Attributs et Méthodes Statistiques]
\begin{verbatim}
#include <algorithm>
#include <vector>

int main() {
    std::vector<int> vec = {5, 2, 9, 1, 5, 6};
    std::sort(vec.begin(), vec.end());

    for (int v : vec) {
        std::cout << v << " ";
    }
    return 0;
}
\end{verbatim}
\end{tcolorbox}

\textbf{Complexité :}
\begin{itemize}
    \item En moyenne : $O(n \log n)$.
\end{itemize}

---

\subsection{ Tri par Fusion (Merge Sort)}
Le tri par fusion divise le tableau en sous-tableaux, les trie récursivement, puis les fusionne.

\textbf{Exemple :}
\begin{tcolorbox}[colframe=blue!50!black, colback=blue!5!white, title=Exemple d'Attributs et Méthodes Statistiques]
\begin{verbatim}
void merge(int arr[], int l, int m, int r) {
    int n1 = m - l + 1;
    int n2 = r - m;

    int L[n1], R[n2];

    for (int i = 0; i < n1; i++) L[i] = arr[l + i];
    for (int i = 0; i < n2; i++) R[i] = arr[m + 1 + i];

    int i = 0, j = 0, k = l;
    while (i < n1 && j < n2) {
        if (L[i] <= R[j]) arr[k++] = L[i++];
        else arr[k++] = R[j++];
    }

    while (i < n1) arr[k++] = L[i++];
    while (j < n2) arr[k++] = R[j++];
}

void mergeSort(int arr[], int l, int r) {
    if (l < r) {
        int m = l + (r - l) / 2;
        mergeSort(arr, l, m);
        mergeSort(arr, m + 1, r);
        merge(arr, l, m, r);
    }
}
\end{verbatim}
\end{tcolorbox}

\textbf{Complexité :}
\begin{itemize}
    \item Pire cas : $O(n \log n)$.
\end{itemize}

---

\subsection{ Comparaison des Algorithmes de Tri}
\begin{center}
\begin{tabular}{|c|c|c|c|}
\hline
\textbf{Algorithme} & \textbf{Meilleur Cas} & \textbf{Pire Cas} & \textbf{Moyenne} \\
\hline
Tri à Bulles & $O(n)$ & $O(n^2)$ & $O(n^2)$ \\
\hline
Tri par Insertion & $O(n)$ & $O(n^2)$ & $O(n^2)$ \\
\hline
\texttt{std::sort} & $O(n \log n)$ & $O(n \log n)$ & $O(n \log n)$ \\
\hline
Tri par Fusion & $O(n \log n)$ & $O(n \log n)$ & $O(n \log n)$ \\
\hline
\end{tabular}
\end{center}
