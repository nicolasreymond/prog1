% Résumé pour 01_Introduction
\section{ Résumé : 01\_Introduction}

\subsection{ Organisation du cours}
Le cours PRG1 utilise :
\begin{itemize}
    \item Supports disponibles sur Moodle.
    \item Basé sur le livre \textit{Programmer en C++ moderne — de C++11 à C++20}.
    \item Langage enseigné : \textbf{C++20}.
    \item Compilation et IDE :
        \begin{itemize}
            \item Compilateur recommandé : \texttt{g++}.
            \item Environnement optionnel : Visual Studio, CLion, etc.
        \end{itemize}
\end{itemize}

\subsection{ Objectifs du cours}
Apprendre les bases de la programmation en C++ :
\begin{itemize}
    \item Comprendre la syntaxe du langage.
    \item Manipuler des types de données, des instructions, des fonctions et des classes.
    \item Développer une logique algorithmique.
    \item Mettre en place des programmes fiables, modulaires et efficaces.
\end{itemize}

\subsection{ Un premier programme en C++}

Voici un exemple de programme simple en C++ :
\begin{tcolorbox}[colframe=blue!50!black, colback=blue!5!white, title=Exemple de Premier Programme]
\begin{verbatim}
#include <iostream>
using namespace std;

int main() {
    cout << "Bonjour, PRG1 !" << endl;
    return 0;
}
\end{verbatim}
\end{tcolorbox}

\textbf{Explications} :
\begin{itemize}
    \item \texttt{\#include <iostream>} : permet d'utiliser \texttt{cout} et \texttt{cin}.
    \item \texttt{using namespace std;} : simplifie l'écriture (on évite d'écrire \texttt{std::cout}).
    \item \texttt{int main()} : point d'entrée du programme.
    \item \texttt{cout << "Bonjour";} : affiche du texte à l'écran.
    \item \texttt{return 0;} : indique que le programme s'est terminé correctement.
\end{itemize}

\subsection{ Étapes de la compilation}
La compilation d'un programme C++ se fait en plusieurs étapes :
\begin{enumerate}
    \item \textbf{Préprocesseur} : gère les directives \texttt{\#include} et \texttt{\#define}.
    \item \textbf{Compilation} : transforme le code source en code machine.
    \item \textbf{Édition de liens} : combine plusieurs fichiers objets pour créer un exécutable.
\end{enumerate}

\subsection{ Erreurs fréquentes}
\begin{itemize}
    \item \textbf{Erreurs de syntaxe} : oublis de \texttt{;} ou d'accolades.
    \item \textbf{Erreurs de compilation} : utilisation de types incompatibles.
    \item \textbf{Erreurs de lien} : fonctions manquantes ou bibliothèques non incluses.
\end{itemize}

\textbf{Exemple d'erreur classique} :
\begin{tcolorbox}[colframe=blue!50!black, colback=blue!5!white, title=Exemple d'Erreur Classique]
\begin{verbatim}
int main() {
    cout << "Bonjour" // Manque le point-virgule
    return 0;
}
\end{verbatim}
\end{tcolorbox}