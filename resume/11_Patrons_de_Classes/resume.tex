% Résumé pour 11_Patrons_de_Classes
\section{ Résumé: 11 Patrons de Classes}

\subsection{ Introduction aux Patrons de Classes}
Un patron de classe (\textbf{template}) permet de créer des classes génériques. Il facilite la réutilisation du code en permettant à une classe de fonctionner avec différents types de données.

\textbf{Syntaxe Générale :}
\begin{tcolorbox}[colframe=blue!50!black, colback=blue!5!white, title=Syntaxe Générale des Templates]
\begin{verbatim}
template<typename T>
class Classe {
    // Définition de la classe avec le type générique T
};
\end{verbatim}
\end{tcolorbox}

---

\subsection{ Exemple Simple de Patron de Classe}
Voici un exemple de classe générique \texttt{Boite} qui peut contenir n'importe quel type de données.

\begin{tcolorbox}[colframe=blue!50!black, colback=blue!5!white, title=Exemple de Patron de Classe]
\begin{verbatim}
#include <iostream>
using namespace std;

template<typename T>
class Boite {
private:
    T valeur;

public:
    Boite(T v) : valeur(v) {}

    T obtenirValeur() {
        return valeur;
    }
};

int main() {
    Boite<int> boiteEntiere(42);       // Contient un entier
    Boite<string> boiteTexte("Bonjour"); // Contient une chaîne

    cout << "Entier : " << boiteEntiere.obtenirValeur() << endl;
    cout << "Texte : " << boiteTexte.obtenirValeur() << endl;
    return 0;
}
\end{verbatim}
\end{tcolorbox}

---

\subsection{ Patrons avec plusieurs Types}
Les patrons peuvent accepter plusieurs paramètres de type.

\textbf{Exemple :}
\begin{tcolorbox}[colframe=blue!50!black, colback=blue!5!white, title=Exemple de Patron avec Plusieurs Types]
\begin{verbatim}
template<typename T, typename U>
class Paire {
private:
    T premier;
    U second;

public:
    Paire(T a, U b) : premier(a), second(b) {}

    void afficher() {
        cout << "Paire : (" << premier << ", " << second << ")" << endl;
    }
};

int main() {
    Paire<int, string> paire(1, "Un");
    paire.afficher(); // Affiche : Paire : (1, Un)
}
\end{verbatim}
\end{tcolorbox}

---

\subsection{ Fonctions Membres Hors de la Classe}
Les définitions de fonctions membres pour les classes génériques doivent aussi utiliser le mot-clé \texttt{template}.

\textbf{Exemple :}
\begin{tcolorbox}[colframe=blue!50!black, colback=blue!5!white, title=Exemple de Fonctions Membres Hors de la Classe]
\begin{verbatim}
template<typename T>
class Boite {
private:
    T valeur;

public:
    Boite(T v);
    T obtenirValeur();
};

template<typename T>
Boite<T>::Boite(T v) : valeur(v) {}

template<typename T>
T Boite<T>::obtenirValeur() {
    return valeur;
}
\end{verbatim}
\end{tcolorbox}

---

\subsection{ Spécialisation des Patrons de Classes}
La spécialisation permet de définir un comportement spécifique pour un type particulier.

\textbf{Exemple :}
\begin{tcolorbox}[colframe=blue!50!black, colback=blue!5!white, title=Exemple de Spécialisation de Patron]
\begin{verbatim}
template<typename T>
class Afficheur {
public:
    void afficher(T valeur) {
        cout << "Valeur générique : " << valeur << endl;
    }
};

// Spécialisation pour string
template<>
class Afficheur<string> {
public:
    void afficher(string valeur) {
        cout << "Chaîne : " << valeur << endl;
    }
};

int main() {
    Afficheur<int> afficheurInt;
    Afficheur<string> afficheurString;

    afficheurInt.afficher(42);           // Affiche : Valeur générique : 42
    afficheurString.afficher("Bonjour"); // Affiche : Chaîne : Bonjour
}
\end{verbatim}
\end{tcolorbox}

---

\subsection{ Patrons avec des Méthodes Statistiques}
Les patrons peuvent aussi contenir des membres statiques. Ces membres sont partagés entre toutes les instances d'un type donné.

\textbf{Exemple :}
\begin{tcolorbox}[colframe=blue!50!black, colback=blue!5!white, title=Exemple de Méthodes Statistiques]
\begin{verbatim}
template<typename T>
class Compteur {
private:
    static int total;

public:
    Compteur() {
        total++;
    }

    static int obtenirTotal() {
        return total;
    }
};

template<typename T>
int Compteur<T>::total = 0;

int main() {
    Compteur<int> c1, c2;
    Compteur<string> c3;

    cout << "Total pour int : " << Compteur<int>::obtenirTotal() << endl; // 2
    cout << "Total pour string : " << Compteur<string>::obtenirTotal() << endl; // 1
}
\end{verbatim}
\end{tcolorbox}

---

\subsection{ Limitations et Avantages des Patrons}
\textbf{Avantages :}
\begin{itemize}
    \item Réduction de la duplication de code.
    \item Flexibilité pour différents types.
    \item Compatible avec les types primitifs et personnalisés.
\end{itemize}

\textbf{Limitations :}
\begin{itemize}
    \item La taille du binaire peut augmenter en raison de la génération de code pour chaque type utilisé.
    \item La gestion des erreurs de compilation peut être complexe.
\end{itemize}
