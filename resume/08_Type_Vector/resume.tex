% Résumé pour 08_Type_Vector
\section{ Résumé : 08\_Type Vector}

\subsection{ Introduction au \texttt{std::vector}}
Le type \texttt{std::vector}, défini dans \texttt{<vector>}, est une structure de données dynamique qui permet de stocker une collection d'éléments.
\begin{itemize}
    \item Similaire à un tableau (\texttt{array}), mais de taille dynamique.
    \item Permet des opérations avancées comme l'ajout ou la suppression d'éléments.
\end{itemize}

\textbf{Exemple :}
\begin{tcolorbox}[colframe=blue!50!black, colback=blue!5!white, title=Exemple de std::vector]
\begin{verbatim}
#include <vector>
using namespace std;

vector<int> nombres = {1, 2, 3, 4, 5};
cout << "Premier élément : " << nombres[0] << endl;
\end{verbatim}
\end{tcolorbox}

---

\subsection{ Création et Initialisation}
\begin{itemize}
    \item Déclaration vide : \texttt{vector<int> vec;}.
    \item Initialisation avec des valeurs : \texttt{vector<int> vec = \{1, 2, 3\};}.
    \item Taille et valeur initiale : \texttt{vector<int> vec(5, 0); // 5 zéros}.
\end{itemize}

\textbf{Exemples :}
\begin{tcolorbox}[colframe=blue!50!black, colback=blue!5!white, title=Exemple de Création et Initialisation]
\begin{verbatim}
vector<int> vide;               // Vecteur vide
vector<int> valeurs = {10, 20}; // Initialisé avec des valeurs
vector<int> zeros(5, 0);        // 5 zéros
\end{verbatim}
\end{tcolorbox}

---

\subsection{ Accès aux Éléments}
\begin{itemize}
    \item \texttt{operator[]} : accès direct.
    \item \texttt{at()} : accès sécurisé (vérifie les limites).
    \item \texttt{front()} : premier élément.
    \item \texttt{back()} : dernier élément.
\end{itemize}

\textbf{Exemple :}
\begin{tcolorbox}[colframe=blue!50!black, colback=blue!5!white, title=Exemple d'Accès aux Éléments]
\begin{verbatim}
vector<int> nombres = {10, 20, 30};
cout << nombres[1];           // 20
cout << nombres.at(2);        // 30
cout << nombres.front();      // 10
cout << nombres.back();       // 30
\end{verbatim}
\end{tcolorbox}

---

\subsection{ Opérations de Base}
\begin{itemize}
    \item \texttt{push\_back(val)} : ajoute un élément à la fin.
    \item \texttt{pop\_back()} : retire le dernier élément.
    \item \texttt{size()} : retourne la taille.
    \item \texttt{clear()} : supprime tous les éléments.
    \item \texttt{empty()} : vérifie si le vecteur est vide.
\end{itemize}

\textbf{Exemple :}
\begin{tcolorbox}[colframe=blue!50!black, colback=blue!5!white, title=Exemple d'Opérations de Base]
\begin{verbatim}
vector<int> vec = {1, 2, 3};
vec.push_back(4);  // {1, 2, 3, 4}
vec.pop_back();    // {1, 2, 3}
cout << vec.size(); // 3
vec.clear();        // Vide le vecteur
\end{verbatim}
\end{tcolorbox}

---

\subsection{ Parcours d'un \texttt{vector}}
\begin{itemize}
    \item Avec une boucle \texttt{for}.
    \item Avec une boucle \texttt{range-based for}.
    \item Avec des itérateurs.
\end{itemize}

\textbf{Exemples :}
\begin{tcolorbox}[colframe=blue!50!black, colback=blue!5!white, title=Exemple de Parcours d'un vector]
\begin{verbatim}
vector<int> nombres = {10, 20, 30};

// Boucle for classique
for (int i = 0; i < nombres.size(); i++) {
    cout << nombres[i] << " ";
}

// Boucle for basée sur la plage
for (int n : nombres) {
    cout << n << " ";
}

// Avec des itérateurs
for (auto it = nombres.begin(); it != nombres.end(); ++it) {
    cout << *it << " ";
}
\end{verbatim}
\end{tcolorbox}

---

\subsection{ Fonctions Avancées}
\begin{itemize}
    \item \texttt{insert(it, val)} : insère un élément à une position donnée.
    \item \texttt{erase(it)} : supprime un élément à une position donnée.
    \item \texttt{resize(n)} : redimensionne le vecteur.
\end{itemize}

\textbf{Exemples :}
\begin{tcolorbox}[colframe=blue!50!black, colback=blue!5!white, title=Exemple de Fonctions Avancées]
\begin{verbatim}
vector<int> vec = {10, 20, 30};
vec.insert(vec.begin() + 1, 15); // {10, 15, 20, 30}
vec.erase(vec.begin() + 2);      // {10, 15, 30}
vec.resize(5, 0);                // {10, 15, 30, 0, 0}
\end{verbatim}
\end{tcolorbox}

---

\subsection{ Comparaison avec les Tableaux}
\begin{center}
    \begin{tabular}{|c|c|}
        \hline
        \textbf{Vecteur (\texttt{vector})} & \textbf{Tableau (\texttt{array})} \\
        \hline
        Taille dynamique & Taille fixe \\
        Mémoire allouée dynamiquement & Mémoire allouée statiquement \\
        Méthodes intégrées (\texttt{push\_back}, etc.) & Pas de méthodes supplémentaires \\
        Plus lent (gestion dynamique) & Plus rapide (gestion fixe) \\
        \hline
    \end{tabular}
\end{center}

---

\subsection{ Exemples Pratiques}
\textbf{Tri d'un vecteur :}
\begin{tcolorbox}[colframe=blue!50!black, colback=blue!5!white, title=Exemple de Tri d'un Vecteur]
\begin{verbatim}
#include <algorithm>
vector<int> vec = {4, 2, 8, 6};
sort(vec.begin(), vec.end()); // {2, 4, 6, 8}
\end{verbatim}
\end{tcolorbox}

\textbf{Rechercher un élément :}
\begin{tcolorbox}[colframe=blue!50!black, colback=blue!5!white, title=Exemple de Recherche d'un Élément]
\begin{verbatim}
#include <algorithm>
vector<int> vec = {10, 20, 30};
if (find(vec.begin(), vec.end(), 20) != vec.end()) {
    cout << "20 trouvé !";
}
\end{verbatim}
\end{tcolorbox}

---

\subsection{ Résumé Visuel des Méthodes}
\begin{center}
    %\includegraphics[width=0.8\textwidth]{../images/vector_methods.png}
\end{center}
