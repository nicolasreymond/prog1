% Résumé pour 09_Surcharge_Fonctions
\section{ Résumé : 09\_Surcharge et Fonctions Génériques}

\subsection{ Qu'est-ce que la Surcharge de Fonctions ?}
La surcharge de fonctions permet de définir plusieurs fonctions ayant le même nom, mais des paramètres différents.
\begin{itemize}
    \item Cela améliore la lisibilité et la flexibilité du code.
    \item Le compilateur choisit la bonne version de la fonction en fonction des arguments fournis.
\end{itemize}

\textbf{Exemple :}
\begin{tcolorbox}[colframe=blue!50!black, colback=blue!5!white, title=Exemple de Surcharge de Fonctions]
\begin{verbatim}
void afficher(int x) {
    cout << "Entier : " << x << endl;
}

void afficher(double x) {
    cout << "Flottant : " << x << endl;
}

void afficher(string x) {
    cout << "Texte : " << x << endl;
}

int main() {
    afficher(42);        // Appelle afficher(int)
    afficher(3.14);      // Appelle afficher(double)
    afficher("Bonjour"); // Appelle afficher(string)
}
\end{verbatim}
\end{tcolorbox}

---

\subsection{ Règles pour la Surcharge de Fonctions}
\begin{itemize}
    \item Les fonctions doivent avoir des signatures différentes (nombre, type ou ordre des paramètres).
    \item Le retour de la fonction n'est pas pris en compte dans la surcharge.
\end{itemize}

\textbf{Exemple Incorrect :}
\begin{tcolorbox}[colframe=blue!50!black, colback=blue!5!white, title=Exemple Incorrect de Surcharge]
\begin{verbatim}
int addition(int a, int b);
double addition(int a, int b); // Erreur : signature identique
\end{verbatim}
\end{tcolorbox}

---

\subsection{ Surcharge d'Opérateurs}
En C++, il est possible de redéfinir les opérateurs pour des types définis par l'utilisateur (par exemple, des classes).
\begin{itemize}
    \item Utilisé pour rendre les objets de classes manipulables comme des types natifs.
    \item Les opérateurs sont redéfinis en utilisant la fonction membre \texttt{operator}.
\end{itemize}

\textbf{Exemple :}
\begin{tcolorbox}[colframe=blue!50!black, colback=blue!5!white, title=Exemple de Surcharge d'Opérateurs]
\begin{verbatim}
class Point {
public:
    int x, y;

    Point(int a, int b) : x(a), y(b) {}

    // Surcharge de l'operateur +=
    Point operator+=(const Point &p) {
        return Point(x + p.x, y + p.y);
    }

    // Surcharge de l'operateur +
    Point operator+(const Point_droit &p, const Point_gauche &q) {
        return Point(p.x + q.x, p.y + q.y);
    }

    void afficher() {
        cout << "(" << x << ", " << y << ")" << endl;
    }
};

int main() {
    Point p1(2, 3), p2(4, 5);
    Point p2 += p1;     // Utilise operator+=
    Point p3 = p1 + p2; // Utilise operator+
    p2.afficher();      // Affiche (6, 8)
}
\end{verbatim}
\end{tcolorbox}

\subsection{ Surcharge de l'opérateur de flux}
Il est possible de surcharger l'opérateur de flux \texttt{<<} pour permettre l'affichage des objets d'une classe. Cette surcharge personnalise le comportement de \texttt{cout} (ou tout autre flux de sortie) pour afficher une représentation spécifique d'un objet.

\subsubsection{ Pourquoi surcharger \texttt{<<} ?}
Par défaut, \texttt{cout} ne sait pas comment afficher les objets définis par l'utilisateur. La surcharge de \texttt{<<} permet de définir une représentation adaptée.

\textbf{Exemple (sans surcharge) :}
\begin{tcolorbox}[colframe=blue!50!black, colback=blue!5!white, title=Exemple sans Surcharge de Flux]
\begin{verbatim}
Point p(2, 3);
cout << p; // Erreur : pas de définition pour afficher un Point
\end{verbatim}
\end{tcolorbox}

En surchargeant \texttt{<<}, nous pouvons afficher un objet comme suit :
\begin{tcolorbox}[colframe=blue!50!black, colback=blue!5!white, title=Exemple avec Surcharge de Flux]
\begin{verbatim}
Point p(2, 3);
cout << p; // Affiche : (2, 3)
\end{verbatim}
\end{tcolorbox}

\subsubsection{ Syntaxe de la surcharge}
Pour surcharger \texttt{<<}, on définit une fonction globale ou amie de la classe :

\begin{tcolorbox}[colframe=blue!50!black, colback=blue!5!white, title=Syntaxe de la Surcharge de Flux]
\begin{verbatim}
ostream& operator<<(ostream &out, const Point &p);
\end{verbatim}
\end{tcolorbox}

\begin{itemize}
    \item \texttt{ostream\&} : La fonction retourne une référence au flux pour permettre le chaînage.
    \item \texttt{ostream \&out} : Flux de sortie (comme \texttt{cout}).
    \item \texttt{const Point \&p} : Objet à afficher, passé par référence constante.
\end{itemize}

\subsubsection{ Exemple complet}
\begin{tcolorbox}[colframe=blue!50!black, colback=blue!5!white, title=Exemple Complet de Surcharge de Flux]
\begin{verbatim}
#include <iostream>
using namespace std;

// Classe Point
class Point {
private:
    int x, y;

public:
    // Constructeur
    Point(int a, int b) : x(a), y(b) {}

    // Surcharge de l'opérateur <<
    friend ostream& operator<<(ostream &out, const Point &p) {
        out << "(" << p.x << ", " << p.y << ")";
        return out; // Permet le chaînage
    }
};

int main() {
    Point p1(2, 3), p2(5, 8);
    cout << p1 << endl; // Affiche : (2, 3)
    cout << p2 << endl; // Affiche : (5, 8)
}
\end{verbatim}
\end{tcolorbox}

\subsubsection{ Explication détaillée}
\begin{itemize}
    \item La fonction \texttt{operator<<} est définie comme amie (\texttt{friend}) pour accéder aux membres privés de \texttt{Point}.
    \item La fonction utilise \texttt{out <<} pour ajouter les données au flux.
    \item La fonction retourne \texttt{out} pour permettre le chaînage :
          \begin{tcolorbox}[colframe=blue!50!black, colback=blue!5!white, title=Exemple de Chaînage de Flux]
\begin{verbatim}
          cout << p1 << " et " << p2;
          // Affiche : (2, 3) et (5, 8)
          \end{verbatim}
\end{tcolorbox}
\end{itemize}

\subsubsection{ Autres opérateurs}
De manière similaire, vous pouvez surcharger l'opérateur \texttt{>>} pour permettre la saisie d'objets.

\textbf{Exemple :}
\begin{tcolorbox}[colframe=blue!50!black, colback=blue!5!white, title=Exemple de Surcharge de l'Opérateur >>]
\begin{verbatim}
istream& operator>>(istream &in, Point &p) {
    in >> p.x >> p.y;
    return in;
}

int main() {
    Point p;
    cin >> p; // Saisir deux entiers pour x et y
    cout << p; // Affiche le point saisi
}
\end{verbatim}
\end{tcolorbox}

\subsubsection{ Résumé}
La surcharge de l'opérateur \texttt{<<} permet de personnaliser l'affichage des objets d'une classe. Elle doit :
\begin{itemize}
    \item Être déclarée comme fonction globale ou amie.
    \item Prendre en paramètre un flux de sortie (\texttt{ostream \&}) et une référence constante à l'objet.
    \item Retourner le flux pour permettre le chaînage.
\end{itemize}

---

\subsection{ Fonctions Génériques avec \texttt{templates}}
Les \textbf{templates} permettent de créer des fonctions génériques qui fonctionnent avec différents types sans redéfinir la fonction.

\textbf{Syntaxe :}
\begin{tcolorbox}[colframe=blue!50!black, colback=blue!5!white, title=Syntaxe des Templates]
\begin{verbatim}
template<typename T>
T addition(T a, T b) {
    return a + b;
}
\end{verbatim}
\end{tcolorbox}

\textbf{Exemple :}
\begin{tcolorbox}[colframe=blue!50!black, colback=blue!5!white, title=Exemple de Template]
\begin{verbatim}
template<typename T>
T addition(T a, T b) {
    return a + b;
}

int main() {
    cout << addition(5, 3) << endl;       // Utilise int
    cout << addition(2.5, 1.5) << endl;  // Utilise double
    return 0;
}
\end{verbatim}
\end{tcolorbox}

---

\subsection{ Surcharge de Fonctions Génériques}
Les \texttt{templates} peuvent être combinés avec des fonctions spécifiques pour des cas particuliers.

\textbf{Exemple :}
\begin{tcolorbox}[colframe=blue!50!black, colback=blue!5!white, title=Exemple de Surcharge de Template]
\begin{verbatim}
template<typename T>
void afficher(T valeur) {
    cout << "Valeur generique : " << valeur << endl;
}

// Surcharge pour string
void afficher(string valeur) {
    cout << "Texte : " << valeur << endl;
}

int main() {
    afficher(42);          // Appelle la fonction generique
    afficher("Bonjour");   // Appelle la surcharge pour string
}
\end{verbatim}
\end{tcolorbox}

---

\subsection{ Templates de Classe}
Les \texttt{templates} peuvent également être utilisés pour créer des classes génériques.

\textbf{Exemple :}
\begin{tcolorbox}[colframe=blue!50!black, colback=blue!5!white, title=Exemple de Template de Classe]
\begin{verbatim}
template<typename T>
class Boite {
private:
    T valeur;
public:
    Boite(T v) : valeur(v) {}
    void afficher() {
        cout << "Valeur : " << valeur << endl;
    }
};

int main() {
    Boite<int> boiteEntiere(42);
    Boite<string> boiteTexte("Bonjour");

    boiteEntiere.afficher(); // Valeur : 42
    boiteTexte.afficher();   // Valeur : Bonjour
}
\end{verbatim}
\end{tcolorbox}

---

\subsection{ Exemples Pratiques}
\textbf{Combinaison de Templates et Surcharge :}
\begin{tcolorbox}[colframe=blue!50!black, colback=blue!5!white, title=Exemple de Combinaison de Templates et Surcharge]
\begin{verbatim}
template<typename T>
T carre(T x) {
    return x * x;
}

// Specialisation pour string
string carre(string x) {
    return x + x; // Repete la chaine
}

int main() {
    cout << carre(5) << endl;        // 25
    cout << carre(3.14) << endl;    // 9.8596
    cout << carre("Hello") << endl; // HelloHello
}
\end{verbatim}
\end{tcolorbox}

