% Résumé pour 04_Fonctions
\section{ Résumé : 04\_Fonctions}

\subsection{ Rôle des Fonctions}
Les fonctions permettent :
\begin{itemize}
    \item De structurer le code pour le rendre plus lisible et réutilisable.
    \item De réaliser des opérations spécifiques (calculs, affichage, etc.).
    \item D'éviter la duplication de code.
\end{itemize}

\textbf{Exemple :}
\begin{tcolorbox}[colframe=blue!50!black, colback=blue!5!white, title=Exemple de Fonction]
\begin{verbatim}
double addition(double a, double b) {
    return a + b;
}

int main() {
    double resultat = addition(3.5, 2.2);
    cout << "Résultat : " << resultat << endl;
    return 0;
}
\end{verbatim}
\end{tcolorbox}

\subsection{ Déclaration et Définition}
\begin{itemize}
    \item Une \textbf{déclaration} informe le compilateur de l'existence de la fonction.
    \item Une \textbf{définition} contient le code de la fonction.
\end{itemize}

\textbf{Exemple :}
\begin{tcolorbox}[colframe=blue!50!black, colback=blue!5!white, title=Exemple de Déclaration et Définition]
\begin{verbatim}
// Déclaration
double addition(double a, double b);

// Définition
double addition(double a, double b) {
    return a + b;
}
\end{verbatim}
\end{tcolorbox}

\subsection{ Arguments et Paramètres}
\begin{itemize}
    \item Les \textbf{paramètres} sont des variables locales définies dans la fonction.
    \item Les \textbf{arguments} sont les valeurs passées à la fonction.
\end{itemize}

\textbf{Exemple :}
\begin{tcolorbox}[colframe=blue!50!black, colback=blue!5!white, title=Exemple d'Arguments et Paramètres]
\begin{verbatim}
void afficherMessage(string message) { // Paramètre : message
    cout << message << endl;
}

int main() {
    afficherMessage("Bonjour, PRG1 !"); // Argument : "Bonjour, PRG1 !"
    return 0;
}
\end{verbatim}
\end{tcolorbox}

\subsection{ Transmission par Valeur ou Référence}
\begin{itemize}
    \item \textbf{Par valeur} : une copie de l'argument est faite (ne modifie pas l'original).
    \item \textbf{Par référence} : permet de modifier la variable originale.
\end{itemize}

\textbf{Exemple :}
\begin{tcolorbox}[colframe=blue!50!black, colback=blue!5!white, title=Exemple de Transmission par Valeur ou Référence]
\begin{verbatim}
// Par valeur
void incrementerValeur(int n) {
    n++;
}

// Par référence
void incrementerReference(int &n) {
    n++;
}
\end{verbatim}
\end{tcolorbox}

\textbf{Différence à l'exécution :}
\begin{tcolorbox}[colframe=blue!50!black, colback=blue!5!white, title=Exemple de Différence à l'Exécution]
\begin{verbatim}
int main() {
    int a = 5;
    incrementerValeur(a);   // a reste 5
    incrementerReference(a); // a devient 6
}
\end{verbatim}
\end{tcolorbox}

\subsection{ Valeurs de Retour}
\begin{itemize}
    \item Les fonctions peuvent retourner une valeur avec \texttt{return}.
    \item Si aucune valeur n'est retournée, le type \texttt{void} est utilisé.
\end{itemize}

\textbf{Exemple :}
\begin{tcolorbox}[colframe=blue!50!black, colback=blue!5!white, title=Exemple de Valeurs de Retour]
\begin{verbatim}
int carre(int x) {
    return x * x;
}

void afficherMessage() {
    cout << "Message sans retour." << endl;
}
\end{verbatim}
\end{tcolorbox}

\subsection{ Arguments par Défaut}
Les arguments peuvent avoir des valeurs par défaut dans leur déclaration.

\textbf{Exemple :}
\begin{tcolorbox}[colframe=blue!50!black, colback=blue!5!white, title=Exemple d'Arguments par Défaut]
\begin{verbatim}
void saluer(string nom = "Utilisateur") {
    cout << "Bonjour, " << nom << " !" << endl;
}

int main() {
    saluer();             // Affiche : Bonjour, Utilisateur !
    saluer("Alice");      // Affiche : Bonjour, Alice !
}
\end{verbatim}
\end{tcolorbox}

\subsection{ Fonction Récursive}
Une fonction peut s'appeler elle-même (récursivité).

\textbf{Exemple : Calcul Factoriel}
\begin{tcolorbox}[colframe=blue!50!black, colback=blue!5!white, title=Exemple de Fonction Récursive]
\begin{verbatim}
int factoriel(int n) {
    if (n <= 1)
        return 1;
    else
        return n * factoriel(n - 1);
}

int main() {
    cout << "Factoriel de 5 : " << factoriel(5) << endl;
    return 0;
}
\end{verbatim}
\end{tcolorbox}

\subsection{ Résumé des Concepts}
\begin{center}
    %\includegraphics[width=0.7\textwidth]{../images/functions_overview.png}
\end{center}
