% Résumé pour 05_Type_String
\section{ Résumé : 05\_Type String}

\subsection{ Introduction au Type \texttt{string}}
Le type \texttt{string} en C++ permet de manipuler facilement des chaînes de caractères.
\begin{itemize}
    \item Défini dans la bibliothèque standard \texttt{<string>}.
    \item Remplace les chaînes de style C (\texttt{char[]}).
\end{itemize}

\textbf{Exemple :}
\begin{tcolorbox}[colframe=blue!50!black, colback=blue!5!white, title=Exemple de Type string]
\begin{verbatim}
#include <string>
using namespace std;

string nom = "Alice";
cout << "Bonjour, " << nom << " !" << endl; // Affiche : Bonjour, Alice !
\end{verbatim}
\end{tcolorbox}

---

\subsection{ Déclaration et Initialisation}
\begin{itemize}
    \item Chaîne vide : \texttt{string ch1; // ch1 est vide}.
    \item Initialisation par une chaîne littérale : \texttt{string ch2 = "Bonjour";}.
    \item Initialisation par répétition : \texttt{string ch3(10, 'x'); // xxxxxxxxxx}.
\end{itemize}

\textbf{Exemple :}
\begin{tcolorbox}[colframe=blue!50!black, colback=blue!5!white, title=Exemple de Déclaration et Initialisation]
\begin{verbatim}
string vide;
string chaine = "Hello";
string repetition(5, '*'); // ***** (5 étoiles)
\end{verbatim}
\end{tcolorbox}

---

\subsection{ Méthodes Utiles pour les Chaînes}
\begin{itemize}
    \item \texttt{size()} ou \texttt{length()} : donne la taille.
    \item \texttt{empty()} : vérifie si la chaîne est vide.
    \item \texttt{at(pos)} ou \texttt{operator[]} : accès au caractère à la position \texttt{pos}.
    \item \texttt{substr(pos, len)} : extrait une sous-chaîne.
    \item \texttt{append()} ou \texttt{+=} : ajoute à la fin.
    \item \texttt{erase(pos, len)} : supprime une portion.
\end{itemize}

\textbf{Exemple :}
\begin{tcolorbox}[colframe=blue!50!black, colback=blue!5!white, title=Exemple de Méthodes Utiles]
\begin{verbatim}
string texte = "Programmation";
cout << "Taille : " << texte.size() << endl;  // Taille : 13
cout << texte.at(0) << texte[1] << endl;      // Pr
cout << texte.substr(0, 7) << endl;           // Program
texte.append(" 1");                           // Ajoute " 1" à la fin
texte.erase(0, 4);                            // Supprime "Prog"
\end{verbatim}
\end{tcolorbox}

---

\subsection{ Concaténation et Comparaison}
\begin{itemize}
    \item \textbf{Concaténation :} Utilisez \texttt{+} ou \texttt{+=}.
    \item \textbf{Comparaison :} Utilisez \texttt{==}, \texttt{!=}, \texttt{<}, \texttt{>}, etc.
\end{itemize}

\textbf{Exemple :}
\begin{tcolorbox}[colframe=blue!50!black, colback=blue!5!white, title=Exemple de Concaténation et Comparaison]
\begin{verbatim}
string a = "Bonjour", b = "Monde";
string c = a + " " + b;  // Bonjour Monde
cout << (a == "Bonjour"); // true
cout << (b > "Monde");    // false
\end{verbatim}
\end{tcolorbox}

---

\subsection{ Lecture et Écriture}
\begin{itemize}
    \item \textbf{Lecture ligne complète :} \texttt{getline(cin, chaine)}.
    \item \textbf{Lecture mot à mot :} \texttt{cin >> chaine}.
\end{itemize}

\textbf{Exemple :}
\begin{tcolorbox}[colframe=blue!50!black, colback=blue!5!white, title=Exemple de Lecture et Écriture]
\begin{verbatim}
string nom, phrase;
cout << "Quel est votre nom ? ";
cin >> nom; // Lit un mot
cin.ignore(); // Ignorer le saut de ligne
cout << "Entrez une phrase complète : ";
getline(cin, phrase);
\end{verbatim}
\end{tcolorbox}

---

\subsection{ Conversion de Numériques en Chaînes et Inversement}
\begin{itemize}
    \item \textbf{Chaîne vers numérique :} Utilisez \texttt{stoi()}, \texttt{stod()}, etc.
    \item \textbf{Numérique vers chaîne :} Utilisez \texttt{to\_string()}.
\end{itemize}

\textbf{Exemple :}
\begin{tcolorbox}[colframe=blue!50!black, colback=blue!5!white, title=Exemple de Conversion]
\begin{verbatim}
string chaine = "123";
int nombre = stoi(chaine);       // Convertit en entier
double reel = stod("12.34");     // Convertit en réel
string texte = to_string(45.67); // "45.67"
\end{verbatim}
\end{tcolorbox}

---

\subsection{ Exemples Pratiques}
\textbf{Manipulation de texte :}
\begin{tcolorbox}[colframe=blue!50!black, colback=blue!5!white, title=Exemple de Manipulation de Texte]
\begin{verbatim}
string phrase = "Programmation C++";
phrase.insert(13, " en");   // Programmation en C++
phrase.replace(0, 12, "Cours"); // Cours en C++
phrase.pop_back();         // Supprime le dernier caractère
cout << phrase;            // Cours en C+
\end{verbatim}
\end{tcolorbox}

\subsection{ Résumé visuel des fonctionnalités}
\begin{center}
    %\includegraphics[width=0.8\textwidth]{../images/string_functions.png}
\end{center}
