% Résumé pour 06_Les_Flux
\section{ Résumé : 06\_Les Flux}

\subsection{ Introduction aux Flux}
Un flux est un canal permettant d'envoyer ou de recevoir des données.
\begin{itemize}
    \item Flux de sortie : envoi des données vers un périphérique (\texttt{cout}).
    \item Flux d'entrée : réception des données depuis un périphérique (\texttt{cin}).
\end{itemize}

\textbf{Exemple :}
\begin{tcolorbox}[colframe=blue!50!black, colback=blue!5!white, title=Exemple de Flux]
\begin{verbatim}
#include <iostream>
using namespace std;

int main() {
    int nombre;
    cout << "Entrez un nombre : ";
    cin >> nombre;
    cout << "Vous avez entré : " << nombre << endl;
    return 0;
}
\end{verbatim}
\end{tcolorbox}

---

\subsection{ Les Flux Prédéfinis}
\begin{itemize}
    \item \texttt{cout} : sortie standard (écran).
    \item \texttt{cin} : entrée standard (clavier).
    \item \texttt{cerr} : sortie standard sans tampon (affichage immédiat des erreurs).
    \item \texttt{clog} : sortie standard avec tampon (messages de log).
\end{itemize}

---

\subsection{ Manipulateurs de Flux}
Les manipulateurs permettent de formater l'affichage dans les flux.

\subsubsection{ Manipulateurs pour les Entiers}
\begin{itemize}
    \item \texttt{dec} : base décimale (par défaut).
    \item \texttt{hex} : base hexadécimale.
    \item \texttt{oct} : base octale.
    \item \texttt{showpos} : affiche le signe \texttt{+} pour les nombres positifs.
\end{itemize}

\textbf{Exemple :}
\begin{tcolorbox}[colframe=blue!50!black, colback=blue!5!white, title=Exemple de Manipulateurs pour les Entiers]
\begin{verbatim}
int nombre = 255;
cout << dec << nombre << endl;  // 255
cout << hex << nombre << endl;  // ff
cout << oct << nombre << endl;  // 377
\end{verbatim}
\end{tcolorbox}

\subsubsection{ Manipulateurs pour les Nombres Réels}
\begin{itemize}
    \item \texttt{fixed} : notation fixe (6 chiffres après la virgule par défaut).
    \item \texttt{scientific} : notation scientifique.
    \item \texttt{setprecision(n)} : spécifie le nombre de chiffres significatifs.
    \item \texttt{showpoint} : force l'affichage des chiffres après la virgule.
\end{itemize}

\textbf{Exemple :}
\begin{tcolorbox}[colframe=blue!50!black, colback=blue!5!white, title=Exemple de Manipulateurs pour les Nombres Réels]
\begin{verbatim}
#include <iomanip>
double pi = 3.14159;
cout << fixed << setprecision(2) << pi << endl;  // 3.14
cout << scientific << pi << endl;               // 3.14e+00
\end{verbatim}
\end{tcolorbox}

---

\subsection{ Lecture au Clavier (\texttt{cin})}
\begin{itemize}
    \item \texttt{cin >> variable} : lit une entrée (ignore les espaces).
    \item \texttt{getline(cin, chaine)} : lit une ligne complète.
\end{itemize}

\textbf{Exemple :}
\begin{tcolorbox}[colframe=blue!50!black, colback=blue!5!white, title=Exemple de Lecture au Clavier]
\begin{verbatim}
#include <string>
string nom;
cout << "Entrez votre nom : ";
getline(cin, nom);
cout << "Bonjour, " << nom << " !" << endl;
\end{verbatim}
\end{tcolorbox}

---

\subsection{ Flux Fichiers}
Pour lire ou écrire des fichiers, on utilise les classes suivantes :
\begin{itemize}
    \item \texttt{ofstream} : flux de sortie vers un fichier.
    \item \texttt{ifstream} : flux d'entrée depuis un fichier.
    \item \texttt{fstream} : flux d'entrée et de sortie.
\end{itemize}

\subsubsection{ Écriture dans un Fichier}
\begin{tcolorbox}[colframe=blue!50!black, colback=blue!5!white, title=Exemple d'Écriture dans un Fichier]
\begin{verbatim}
#include <fstream>
ofstream fichier("sortie.txt");
fichier << "Hello, fichier !" << endl;
fichier.close();
\end{verbatim}
\end{tcolorbox}

\subsubsection{ Lecture depuis un Fichier}
\begin{tcolorbox}[colframe=blue!50!black, colback=blue!5!white, title=Exemple de Lecture depuis un Fichier]
\begin{verbatim}
#include <fstream>
ifstream fichier("entree.txt");
string ligne;
while (getline(fichier, ligne)) {
    cout << ligne << endl;
}
fichier.close();
\end{verbatim}
\end{tcolorbox}

---

\subsection{ Manipulation Avancée des Fichiers}
\begin{itemize}
    \item \texttt{seekg(pos)} : déplace le pointeur de lecture.
    \item \texttt{seekp(pos)} : déplace le pointeur d'écriture.
    \item \texttt{tellg()} : donne la position actuelle du pointeur de lecture.
    \item \texttt{tellp()} : donne la position actuelle du pointeur d'écriture.
\end{itemize}

\textbf{Exemple : Lecture Inversée d'un Fichier :}
\begin{tcolorbox}[colframe=blue!50!black, colback=blue!5!white, title=Exemple de Lecture Inversée d'un Fichier]
\begin{verbatim}
#include <fstream>
ifstream fichier("texte.txt", ios::ate); // Ouvert en mode "fin de fichier"
streampos taille = fichier.tellg();     // Taille du fichier
for (int i = taille - 1; i >= 0; --i) {
    fichier.seekg(i);
    char c;
    fichier.get(c);
    cout << c;
}
fichier.close();
\end{verbatim}
\end{tcolorbox}

---

\subsection{ Résumé Visuel des Flux}
\begin{center}
    %\includegraphics[width=0.8\textwidth]{../images/flux_summary.png}
\end{center}
