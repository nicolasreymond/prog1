% Résumé pour 03_Instructions_Controle
\section{ Résumé : 03\_Instructions de Contrôle}

\subsection{ Structure Séquentielle}
Par défaut, les instructions d'un programme sont exécutées dans l'ordre, de haut en bas.

\textbf{Exemple :}
\begin{tcolorbox}[colframe=blue!50!black, colback=blue!5!white, title=Exemple de Structure Séquentielle]
\begin{verbatim}
int a = 5;
cout << "Valeur de a : " << a << endl;
// Résultat : Valeur de a : 5
\end{verbatim}
\end{tcolorbox}

\subsection{ Structures Conditionnelles}
Les structures conditionnelles permettent de prendre des décisions dans le programme.

\subsubsection{ Instruction \texttt{if-else}}
\begin{itemize}
    \item Exécute un bloc d'instructions si une condition est vraie.
\end{itemize}

\textbf{Syntaxe :}
\begin{tcolorbox}[colframe=blue!50!black, colback=blue!5!white, title=Syntaxe de l'Instruction if-else]
\begin{verbatim}
if (condition) {
    // Instructions si condition est vraie
} else {
    // Instructions si condition est fausse
}
\end{verbatim}
\end{tcolorbox}

\textbf{Exemple :}
\begin{tcolorbox}[colframe=blue!50!black, colback=blue!5!white, title=Exemple de l'Instruction if-else]
\begin{verbatim}
int age = 20;
if (age >= 18) {
    cout << "Vous êtes majeur." << endl;
} else {
    cout << "Vous êtes mineur." << endl;
}
\end{verbatim}
\end{tcolorbox}

\subsubsection{ Instruction \texttt{switch}}
\begin{itemize}
    \item Permet de choisir parmi plusieurs options.
    \item Utilise des \texttt{case} et un \texttt{default}.
\end{itemize}

\textbf{Syntaxe :}
\begin{tcolorbox}[colframe=blue!50!black, colback=blue!5!white, title=Syntaxe de l'Instruction switch]
\begin{verbatim}
switch (variable) {
    case valeur1:
        // Instructions
        break;
    case valeur2:
        // Instructions
        break;
    default:
        // Instructions par défaut
}
\end{verbatim}
\end{tcolorbox}

\textbf{Exemple :}
\begin{tcolorbox}[colframe=blue!50!black, colback=blue!5!white, title=Exemple de l'Instruction switch]
\begin{verbatim}
int jour = 3;
switch (jour) {
    case 1:
        cout << "Lundi" << endl;
        break;
    case 2:
        cout << "Mardi" << endl;
        break;
    default:
        cout << "Autre jour" << endl;
}
\end{verbatim}
\end{tcolorbox}

\subsection{ Structures Itératives}
Les boucles permettent de répéter un bloc d'instructions plusieurs fois.

\subsubsection{ Boucle \texttt{while}}
\begin{itemize}
    \item Exécute tant qu'une condition est vraie.
\end{itemize}

\textbf{Syntaxe :}
\begin{tcolorbox}[colframe=blue!50!black, colback=blue!5!white, title=Syntaxe de la Boucle while]
\begin{verbatim}
while (condition) {
    // Instructions
}
\end{verbatim}
\end{tcolorbox}

\textbf{Exemple :}
\begin{tcolorbox}[colframe=blue!50!black, colback=blue!5!white, title=Exemple de la Boucle while]
\begin{verbatim}
int i = 0;
while (i < 5) {
    cout << i << endl;
    i++;
}
\end{verbatim}
\end{tcolorbox}

\subsubsection{ Boucle \texttt{do-while}}
\begin{itemize}
    \item Exécute au moins une fois, puis tant que la condition est vraie.
\end{itemize}

\textbf{Syntaxe :}
\begin{tcolorbox}[colframe=blue!50!black, colback=blue!5!white, title=Syntaxe de la Boucle do-while]
\begin{verbatim}
do {
    // Instructions
} while (condition);
\end{verbatim}
\end{tcolorbox}

\textbf{Exemple :}
\begin{tcolorbox}[colframe=blue!50!black, colback=blue!5!white, title=Exemple de la Boucle do-while]
\begin{verbatim}
int i = 0;
do {
    cout << i << endl;
    i++;
} while (i < 5);
\end{verbatim}
\end{tcolorbox}

\subsubsection{ Boucle \texttt{for}}
\begin{itemize}
    \item Boucle compacte avec initialisation, condition et incrémentation.
\end{itemize}

\textbf{Syntaxe :}
\begin{tcolorbox}[colframe=blue!50!black, colback=blue!5!white, title=Syntaxe de la Boucle for]
\begin{verbatim}
for (initialisation; condition; incrément) {
    // Instructions
}
\end{verbatim}
\end{tcolorbox}

\textbf{Exemple :}
\begin{tcolorbox}[colframe=blue!50!black, colback=blue!5!white, title=Exemple de la Boucle for]
\begin{verbatim}
for (int i = 0; i < 5; i++) {
    cout << i << endl;
}
\end{verbatim}
\end{tcolorbox}

\subsection{ Instructions de Saut}
\begin{itemize}
    \item \textbf{\texttt{break}} : Sort de la boucle ou du \texttt{switch}.
    \item \textbf{\texttt{continue}} : Passe directement à l'itération suivante.
    \item \textbf{\texttt{goto}} : Saute à une étiquette (peu recommandé).
\end{itemize}

\textbf{Exemple d'utilisation de \texttt{break} :}
\begin{tcolorbox}[colframe=blue!50!black, colback=blue!5!white, title=Exemple d'Utilisation de break]
\begin{verbatim}
for (int i = 0; i < 10; i++) {
    if (i == 5) break; // Sort de la boucle si i vaut 5
    cout << i << endl;
}
\end{verbatim}
\end{tcolorbox}

\textbf{Exemple d'utilisation de \texttt{continue} :}
\begin{tcolorbox}[colframe=blue!50!black, colback=blue!5!white, title=Exemple d'Utilisation de continue]
\begin{verbatim}
for (int i = 0; i < 5; i++) {
    if (i == 2) continue; // Passe directement à i = 3
    cout << i << endl;
}
\end{verbatim}
\end{tcolorbox}

\subsection{ Résumé visuel des instructions de contrôle}
\begin{center}
    %\includegraphics[width=0.8\textwidth]{../images/flow_control.png}
\end{center}
